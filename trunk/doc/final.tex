\documentclass{scrartcl}
\usepackage[utf8]{inputenc}
\usepackage[T1]{fontenc}
\usepackage[ngerman]{babel}
\usepackage{amsmath}
\usepackage{graphicx}
\usepackage{multirow}
\usepackage{cite}
\usepackage{hyperref}

\usepackage{geometry}
\geometry{a4paper,left=25mm,right=25mm, top=2cm, bottom=2cm} 

\title{"The Boiler Makers"}
\author{Camillo Dell'mour 0628020 e0628020@student.tuwien.ac.at\\ Thomas Moerwald 0255334 moerwald@acin.tuwien.ac.at}
\begin{document}
\maketitle
%\tableofcontents

\section{Content}
In a small factory of the 18th century, there's a lot going on. Fire is burning to heat the water in a boiler. Steam is exhausted through several valves and leaks of the piston chamber. Particles of hot glowing metal are hurtling through the environment which is sparsely illuminated by several light bulbs.

\section{Effekts und details}
\subsection*{Implemented}
\begin{itemize}
	\item Kinematic animation: \href{http://en.wikipedia.org/wiki/File:Steam_engine_in_action.gif}{Steam engine in action}
	\item Bump mapping especially on rusty metal and wooden surfaces \cite{moeller1999realtime, dietrich2000hardware}
	\item Shading (Phong) \cite{phong1975illumination}
	\item Shading - Ambient occlusion (offline) \cite{hoberock2000high}
	\item Shadowing and self-shadowing \cite{nagy2000realtime, king2000ground}
	\item Steam exhausted by the engine \cite{van2000building}
	\item Particle system simulating the effect of sparks produced by the hammer falling on the ambos \cite{van2000building}
\end{itemize}
\subsection*{Future work}
\begin{itemize}
	\item Refraction mapping of water in the compensation tank for the boiler \cite{vlachos2000refraction}
	\item Motion blur of fast moving devices such as the falling hammer, the spinning fly wheel and regulator \cite{rosado2007motion}
	\item Fire beneath the boiler and smoke produces by it: \href{http://graphics.ethz.ch/teaching/former/imagesynthesis_06/miniprojects/p3/index.html}{Rendering smoke and fire in real-time}
	\item Dynamic illumination caused by flickering light bulb
	\item Detail maps simulating dirt and dust, especially on upper surfaces
\end{itemize}

%\clearpage 
%\addcontentsline{toc}{chapter}{\bibname}
%\bibliographystyle{babplain} 
\bibliography{literatur}     %BibTeX-Datei literatur.bib
\bibliographystyle{plain} 
\end{document}
